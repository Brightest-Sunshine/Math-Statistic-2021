\documentclass[a4paper]{article}
\usepackage[utf8]{inputenc}
\usepackage[12pt]{extsizes}
\usepackage{amsmath,amsthm,amssymb}
\usepackage[hidelinks]{hyperref} 
\usepackage[warn]{mathtext}
\usepackage[T1,T2A]{fontenc}
\usepackage[utf8]{inputenc}
\usepackage[english,russian]{babel}
\usepackage{tocloft}
\linespread{1.5}
\usepackage{indentfirst}
\usepackage{setspace}
%\полуторный интервал
\onehalfspacing

\newcommand{\RomanNumeralCaps}[1]
    {\MakeUppercase{\romannumeral #1}}

\usepackage{amssymb}

\usepackage{graphicx, float}
\graphicspath{{pictures/}}
\DeclareGraphicsExtensions{.pdf,.png,.jpg}
\usepackage[left=25mm,right=1cm,
    top=2cm,bottom=20mm,bindingoffset=0cm]{geometry}
\renewcommand{\cftsecleader}{\cftdotfill{\cftdotsep}}

\addto\captionsrussian{\renewcommand{\contentsname}{СОДЕРЖАНИЕ}}
\addto\captionsrussian{\renewcommand{\listtablename}{СПИСОК ТАБЛИЦ}}

\usepackage{fancyhdr}
\usepackage[nottoc]{tocbibind}

\fancypagestyle{plain}{
\fancyhf{}
\renewcommand{\headrulewidth}{0pt}
\fancyhead[R]{\thepage}
}

\usepackage{blindtext}
\pagestyle{myheadings}
\usepackage{hyperref}

\begin{document}
\begin{titlepage}
  \begin{center}
    \large
    Санкт-Петербургский политехнический университет Петра Великого
    
    Институт прикладной математики и механики
    
    \textbf{Высшая школа прикладной математики и вычислительной физики}
    \vfill
    \textsc{\textbf{\large{Отчёт по лабораторной работе №2}}}\\[5mm]
     по дисциплине\\ <<Математическая статистика>>\\
\end{center}

\vfill

\begin{tabular}{l p{140} l}
Выполнила студентка \\группы 3630102/80401 && Мамаева Анастасия Сергеевна \\
\\
Проверил\\Доцент, к.ф.-м.н.& \hspace{0pt} &   Баженов Александр Николаевич \\\\
\end{tabular}

\hfill \break
\hfill \break
\begin{center} Санкт-Петербург \\2021 \end{center}
\thispagestyle{empty}
\end{titlepage}
\newpage
\newpage
\begin{center}
    \setcounter{page}{2}
    \tableofcontents
\end{center}
\newpage
\begin{center}
    \setcounter{page}{3}
    \listoftables
\end{center}

\newpage
\section {Постановка задачи}
\noindent Сгенерировать выборки размером 10, 100 и 1000 элементов.
Для каждой выборки вычислить следующие статистические характеристики положения данных: $\overline{x}, med x, z_R, z_Q, z_{tr}.$ Повторить такие вычисления 1000 раз для каждой выборки и найти среднее характеристик положения и их квадратов:
\begin{equation}
	E(z) = \overline{z}
\end{equation}
Вычислить оценку дисперсии по формуле:
\begin{equation}
	D(z) = \overline{z^2} - \overline{z}^2
\end{equation}
Представить полученные данные в виде таблиц.

\section {Теория}
\subsection{Распределения}
	\begin{itemize}
		\item Нормальное распределение \begin{equation}
										  N(x, 0, 1) = \frac{1}{\sqrt{2\pi}}e^{\frac{-x^2}{2}} \label{norm} 
									   \end{equation}
		\item Распределение Коши \begin{equation}
									C(x, 0, 1) = \frac{1}{\pi}\frac{1}{x^2+1} \label{koshi}
								 \end{equation} 
		\item Распределение Лапласа \begin{equation}
									   L(x, 0, \frac{1}{\sqrt{2}}) = \frac{1}{\sqrt{2}}e^{-\sqrt{2}|x|} \label{laplace} 
									\end{equation}
		\item Распределение Пуассона \begin{equation}
										P(k, 10) = \frac{10^k}{k!}e^{-10}\label{puasson}
									 \end{equation}
		\item Равномерное распределение \begin{equation}
				U(x, -\sqrt{3}, \sqrt{3}) =
				\begin{cases}
					\frac{1}{2\sqrt{3}} &\text{$при |x|\leq \sqrt{3}$}\\
					0 &\text{$при |x|>\sqrt{3}$}
				\end{cases}
				\label{uni} 
			\end{equation}
	\end{itemize}

\subsection{Вариационный ряд}
	\noindent Вариационным рядом называется последовательность элементов выборки, расположенных в неубывающем порядке. Одинаковые элементы повторяются.
	Запись вариационного ряда: $x_{(1)}, x_{(2)}, \ldots, x_{(n)}$.
	Элементы вариационного ряда $x_{(i)} (i = 1, 2, \ldots, n)$ называются порядковыми статистиками.
	
	\subsection{Выборочные числовые характеристики}
	\noindent С помощью выборки образуются её числовые характеристики. Это числовые характеристики дискретной случайной величины $X^{*}$, принимающей выборочные значения $x_{(1)}, x_{(2)}, \ldots, x_{(n)}$.
	
	\subsubsection{Характеристики положения}
	\begin{itemize}
		\item Выборочное среднее \begin{equation}
									 \overline{x} = \frac{1}{n}\sum_{i=1}^{n}{x_i}
								\end{equation}
		\item Выборочная медиана \begin{equation}
								 	med x = \begin{cases}
											 	x_{(l+1)} &\text{$ n=2l+1$}\\
											 	\frac{x_{(l)} + x_{(l+1)}}{2} &\text{$ n=2l$}
								 			\end{cases}
								 \end{equation}
		\item Полусумма экстремальных выборочных элементов \begin{equation}
														       z_R = \frac{x_{(1)} + x_{(n)}}{2}
														   \end{equation}
		\item Полусумма квартилей \newline Выборочная квартиль $z_p$ порядка $p$ определяется формулой \begin{equation}
				 	z_p = \begin{cases}
		             	  	x_{([np]+1)} &\text{$np - $дробное}\\
		      			    x_{(np)}&\text{$np - $целое}
		      			  \end{cases}
				 \end{equation}
				 Полусумма квартилей \begin{equation}
				 					 	z_Q = \frac{z_{1/4} + z_{3/4}}{2}
				 					 \end{equation}
		\item Усечённое среднее\begin{equation}
							   		z_{tr} = \frac{1}{n-2r}\sum_{i=r+1}^{n-r}{x_{(i)}}, r\approx\frac{n}{4}	   	\end{equation}
	\end{itemize}

	\subsubsection{Характеристики рассеяния}
	Выборочная дисперсия
	\begin{equation}
		D = \frac{1}{n}\sum_{i=1}^{n}{(x_i-\overline{x})^2}
	\end{equation}
	
\section {Программная реализация}
\noindent Лабораторная работа выполнена на языке Python вресии 3.7 в среде разработки JupyterLab. Использовались дополнительные библиотеки:\\ \newline
1. scipy\newline
2. numpy\newline
\\
В приложении находится ссылка на GitHub репозиторий с исходныи кодом.

\section {Результаты} 
\subsection{Характеристики положения и рассеяния}
	\begin{table}[H]
		\centering
		\begin{tabular}[t]{lrrrrr}
			\hline
			Characteristic   &      Mean &    Median &       $z_R$ &      $z_Q$ &      $z_{tr}$ \\
			\hline
			Normal E(z) 10   &  -0.000003 & -0.008031 & 0.036451 & -0.021004 & -0.012014\\
			Normal D(z) 10   &  0.101239 & 0.091199 & 0.480519 & 0.475769 & 0.172801\\
			Normal E(z) 100  & -0.002052 & -0.004684 & 0.012205 & -0.011495 & -0.008917\\
			Normal D(z) 100  & 0.05601 & 0.050246 & 0.478019 & 0.50419 & 0.097187\\
			Normal E(z) 1000 & -0.001219 & -0.003105 & 0.006667 & -0.010076 & -0.006475\\
			Normal D(z) 1000 &  0.037681 & 0.033809 & 0.475955 & 0.497166 & 0.065463\\
			\hline
		\end{tabular}
		\caption{Нормальное распределение \eqref{norm}}
		\label{tab:normal}
	\end{table}
	
	\begin{table}[H]
	\centering
		\begin{tabular}[t]{lrrrrr}
			\hline
			Characteristic   &        Mean &    Median &            $z_R$ &       $z_Q$ &      $z_{tr}$ \\
			\hline
			Cauchy E(z) 10   &   -1.787828 & -0.025951 & -5.130831 & -3.948137 & -2.645886\\
			Cauchy D(z) 10   &  4390.971 & 0.403047 & 71558.563 & 21478.02 & 2724.385 \\
			Cauchy E(z) 100  &   -2.946291 & -0.011733 & -2.919987 & 7.274195 & -5.064231 \\
			Cauchy D(z) 100  & 11499.485 & 0.214055 & 37287.433 & 194370.44 & 37068.767  \\
			Cauchy E(z) 1000 &   -2.272925 & -0.007192 & -1.740728 & 4.70099 & -4.116761 \\
			Cauchy D(z) 1000 & 7794.082 & 0.143556 & 25043.95 & 129732.73 & 25183.86 \\
			\hline
		\end{tabular}
	\caption{Распределение Коши \eqref{koshi}}
	\label{tab:cauchy}
	\end{table}

\begin{table}[H]
	\centering
		\begin{tabular}[t]{lrrrrr}
			\hline
			Characteristic    &      Mean &    Median &       $z_R$ &       $z_Q$ &      $z_{tr}$ \\
			\hline
			Laplace E(z) 10   &  0.022501 & 0.011001 & 0.011483 & 0.023003 & 0.027397 \\
			Laplace D(z) 10   &  0.103582 & 0.077859 & 0.490929 & 0.47279 & 0.169248 \\
			Laplace E(z) 100  &  0.012271 & 0.005379 & 0.021852 & 0.021828 & 0.015041 \\
			Laplace D(z) 100  &  0.057048 & 0.041891 & 0.492694 & 0.492542 & 0.095815 \\
			Laplace E(z) 1000 &  0.008403 & 0.003924 & 0.022277 & 0.022505 & 0.010596 \\
			Laplace D(z) 1000 &  0.038374 & 0.028103 & 0.487406 & 0.505289 & 0.064555 \\
			\hline
		\end{tabular}
		\caption{Распределение Лапласа \eqref{laplace}}
		\label{tab:laplace}
	\end{table}

\begin{table}[H]
		\centering
		\begin{tabular}[t]{lrrrrr}
			\hline
			Characteristic    &      Mean &   Median &       $z_R$ &      $z_Q$ &     $z_{tr}$ \\
			\hline
			Poisson E(z) 10   & 9.979 & 9.8525 & 9.9655 & 10.044 & 9.997167     \\
			Poisson D(z) 10   &  0.975799 & 1.373994 & 4.68356 & 5.018564 & 1.779575  \\
			Poisson E(z) 100  & 9.98473 & 9.8535 & 9.9615 & 10.0425 & 9.986043  \\
			Poisson D(z) 100  &  0.534271 & 0.786788 & 4.841018 & 5.073944 & 0.982735 \\
			Poisson E(z) 1000 & 9.990781 & 9.902 & 9.955 & 10.057 & 9.99035\\
			Poisson D(z) 1000 &  0.359408 & 0.529563 & 4.769308 & 5.025751 & 0.661672 \\
			\hline
		\end{tabular}
		
		\caption{Распределение Пуассона \eqref{puasson}}
		\label{tab:poisson}
	\end{table}

\begin{table}[H]
		\centering
		\begin{tabular}[t]{lrrrrr}
			\hline
			Characteristic    &      Mean &    Median &       $z_{R}$ &       $z_Q$ &      $z_{tr}$ \\
			\hline
			Uniform E(z) 10   &  -0.003058 & -0.008189 & -0.008308 & -0.058391 & -0.000635 \\
			Uniform D(z) 10   &  0.10312 & 0.240584 & 0.509498 & 0.516179 & 0.1717448 \\
			Uniform E(z) 100  &  -0.001261 & -0.005273 & -0.013241 & -0.026719 & -0.001152 \\
			Uniform D(z) 100  &  0.056234 & 0.134553 & 0.506669 & 0.513444 & 0.095636 \\
			Uniform E(z) 1000 & -0.000621 & -0.003033 & -0.005643 & -0.017729 & -0.000709  \\
			Uniform D(z) 1000 &  0.037829 & 0.090714 & 0.499291 & 0.501763 & 0.064392 \\
			\hline
		\end{tabular}
		\caption{Равномерное распределение \eqref{uni}}
		\label{tab:uniform}
	\end{table}

\section {Обсуждение} 

\noindent Исходя из данных, приведенных в таблицах, можно судить о том, что дисперсия характеристик рассеяния для распределения Коши является некой аномалией: значения слишком большие даже при увеличении размера выборки - понятно, что это результат выбросов, которые мы могли наблюдать в результатах предыдущего задания.

\section {Приложение}
\noindent Код программы GitHub URL:\\
\newline https://github.com/Brightest-Sunshine/Math-Statistic-2021/blob/main/Lab2/Lab2.ipynb

\end{document}